%\clearpage
%\addcontentsline{toc}{chapter}{Appendices}
\begin{appendices}
\chapter{\label{appendix}(AUC-ROC Curve in Machine Learning)}
The AUC-ROC curve helps us visualize how well our machine learning classifier is performing.
\section{What is the AUC-ROC curve?}
The Receiver Operator Characteristic (ROC) curve is an evaluation metric for binary classification problems. It is a probability curve that plots the TPR against FPR at various threshold values and essentially separates the ‘signal’ from the ‘noise’. The Area Under the Curve (AUC) is the measure of the ability of a classifier to distinguish between classes and is used as a summary of the ROC curve.\\
The higher the AUC, the better the performance of the model at distinguishing between the positive and negative classes.\\
When AUC = 1, then the classifier is able to perfectly distinguish between all the Positive and the Negative class points correctly. If, however, the AUC had been 0, then the classifier would be predicting all Negatives as Positives, and all Positives as Negatives.\\
https://www.analyticsvidhya.com/blog/2020/06/auc-roc-curve-machine-learning/
\end{appendices}


\setcounter{equation}{0}
\setcounter{table}{0}
\setcounter{figure}{0}
%\baselineskip 24pt


    



